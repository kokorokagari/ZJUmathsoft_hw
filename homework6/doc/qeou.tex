%!Tex Program = xelatex
%\documentclass[a4paper]{article}
\documentclass[a4paper]{ctexart}

\usepackage[utf8]{inputenc}%告诉 LaTeX TeX文档的输入(文件)编码(为utf-8)
\usepackage{amsmath}%数学公式相关
\usepackage{amssymb}%更多的数学符号
\usepackage{caption}%控制题目/说明文字的样式
\usepackage{graphicx, subfig}%插入图片
\usepackage{ctex}%中文宏包
\usepackage{float}%强行保留图片位置用
%\usepackage[T1]{fontenc}%可以调用各种编码字体,并可有多重选项,其中最后的选项将成为系统默认编码
%\usepackage{textcomp}%提供了许多 TS1 编码的可直接在文本中使用的常见数理单位和货币符号
\usepackage{url}%插入网址
\usepackage{hyperref}%超链接

\usepackage{tikz}%TiKZ绘图
\usetikzlibrary{arrows,shapes,chains}
\usetikzlibrary{graphs, positioning, quotes, shapes.geometric}
\usepackage{array}

\usepackage{listings}%更好看的代码
\usepackage{xltxtra}%listings中文支持
\usepackage{color}%listings字体颜色相关
\definecolor{dkgreen}{rgb}{0,0.6,0}
\definecolor{gray}{rgb}{0.5,0.5,0.5}
\definecolor{mauve}{rgb}{0.58,0,0.82}

\lstset{frame=tb,%listings代码区各种设置
	language=Python,
	aboveskip=3mm,
	belowskip=3mm,
	showstringspaces=false,
	columns=flexible,
	basicstyle={\small\ttfamily},
	numbers=left,%设置行号位置none不显示行号
	%numberstyle=\tiny\courier, %设置行号大小
	numberstyle=\tiny\color{gray},
	keywordstyle=\color{blue},
	commentstyle=\color{dkgreen},
	stringstyle=\color{mauve},
	breaklines=true,
	breakatwhitespace=true,
	escapeinside=``,%逃逸字符(1左面的键),用于显示中文例如在代码中`中文...`
	tabsize=4,
	extendedchars=false %解决代码跨页时,章节标题,页眉等汉字不显示的问题
}

\hypersetup{hypertex=true,%超链接设置
	colorlinks=true,
	linkcolor=blue,
	anchorcolor=blue,
	citecolor=blue}
%%%%%%%%%%%%%%%%%%%%%%%%%%%%%%%%%%%%%%%%%%%%%%%%%%%%%%%%%%%%%%%%%%%%%
%% Place any additional macros here.  Please use \newcommand* where
%% possible, and avoid layout-changing macros (which are not used
%% when typesetting).
%%%%%%%%%%%%%%%%%%%%%%%%%%%%%%%%%%%%%%%%%%%%%%%%%%%%%%%%%%%%%%%%%%%%%
%\newcommand*\mycommand[1]{\texttt{\emph{#1}}}

%\setmainfont[Mapping=tex-text]{AR PL UMing CN:style=Light}
%\setmainfont[Mapping=tex-text]{AR PL UKai CN:style=Book}
%\setmainfont[Mapping=tex-text]{WenQuanYi Zen Hei:style=Regular}
%\setmainfont[Mapping=tex-text]{WenQuanYi Zen Hei Sharp:style=Regular}
%\setmainfont[Mapping=tex-text]{AR PL KaitiM GB:style=Regular} 
%\setmainfont[Mapping=tex-text]{AR PL SungtiL GB:style=Regular} 
%\setmainfont[Mapping=tex-text]{WenQuanYi Zen Hei Mono:style=Regula}

\author{吴泓鹰\\数学与应用数学(强基计划)\quad3210101890}
\title{\textbf{一元二次方程在实数域上的求解}}
%\date{}

\begin{document}
\maketitle
总所周之,在实数上一元二次方程的求解是小学二年级就学会的内容,甚至于对于小学生而言这太过幼稚,相比之下,对于大学生来是就刚刚好。因此我准备在这里介绍如何在实数域求解一元二次方程并且使用TiKZ绘制算法流程图、使用Gnuplot绘制方程解的三种情况情况示意图的。(绘图参考了\cite{tikz}\cite{gnuplot})

\CTEXsetup[format={\Large\bfseries}]{section}%使其后的标题左对齐
\section{数学原理}	
求解一元二次方程也就是求满足$ax^2+bx+c=0$的$x$的值,这里$a,b,c,x\in\mathbb{R}$,让我们采访一下不同人是怎么求$x$的吧:\cite{xiantong}
\begin{enumerate}
\item  \textbf{小学生}:作为一名优秀的小学生,老师告诉过我,解一元二次方程只需要配方,移项,开根号,移项化简,这样就能得到最终的解啦。
	$$
	ax^2+bx+c=0\Rightarrow a(x+\frac{b}{2a})^2+c-\frac{b^2}{4a}=0
	\Rightarrow (x+\frac{b}{2a})^2=\frac{b^2-4ac}{4a^2}
	$$
	\begin{equation}\label{root}
	\Rightarrow x+\frac{b}{2a}=\pm\sqrt{\frac{b^2-4ac}{4a^2}}
	\Rightarrow x=\frac{-b\pm\sqrt{b^2-4ac}}{2a}
	\end{equation}
\item \textbf{初中生}:作为一名优秀的初中生,我不知道什么配方法,只知道一元二次方程(可能)有两个根$x_1,x_2$,即$x=x_1$和$x_2$,因此有:
	$$
	x=\frac{x_1+x_2}{2}\pm\frac{|x_1-x_2|}{2}=\frac{x_1+x_2}{2}\pm\frac{\sqrt{(x_1+x_2)^2-4x_1 x_2}}{2}
	$$
	进而根据韦达定理:$x_1+x_2=-\frac{b}{a},x_1 x_2=\frac{c}{a}$,就可以把上式化为\ref{root}式中的结果啦。
\item \textbf{高中生}:作为一名优秀的高中生,什么配方、韦达定理我都不会,只会求导、积分,所以有:
	$$
	f(x)=ax^2+bx+c\Rightarrow f'(x)=2ax+b\Rightarrow f(x)=\int f'(x)dx=\frac{(2ax+b)^2}{4a}+C
	$$
	$$
	\Rightarrow C=c-\frac{b^2}{4a},\int f'(x)dx=0\Rightarrow \frac{(2ax+b)^2}{4a}=\frac{b^2}{4a}-c
	$$
	就可以类似地得到\ref{root}式中的结果啦。
\item \textbf{大学生}:作为一名优秀的大学生,配方、韦达、求导、积分什么的我听都没听说过,但我知道一元五次方程没有根式解,这说明群$S_5$不可解,即不存在链$\{e\}\vartriangleleft...\vartriangleleft S_5$使得每一个群都是其后面一个群的正规子群,且相邻群做商群是一个abel群,这是因为$S_5$的正规子群$A_5$的正规子群恒为$A_5$。而对于一元二次方程,群$S_2\cong \mathbb{Z}_2$,而$\mathbb{Z}_2$是个able群,故$S_2$是个able群,又$S_2/\{e\}\cong S_2$,这表明一元二次方程有求根公式,但怎么得到这个求根公式,这超出了我的能力范围。
\end{enumerate}
\section{用TiKZ绘制算法流程图}	
\begin{figure}[H]
\tikzstyle{io} = [trapezium, trapezium left angle = 60, trapezium right angle = 120, minimum width = 1cm, text centered, draw = black, fill = white, align=center]
\centering
\begin{tikzpicture}[node distance=10pt]
		\node[draw, rounded corners](start){开始};
		\node[draw, io , below=of start](step 1){输入$a,b,c$};
		\node[draw, below=of step 1](step 2){计算$\Delta=b^2-4ac$};
		\node[draw, diamond, aspect=2, below=of step 2](choice1){$\Delta\ge0$};
		\node[draw, right=30pt of choice1](choice2){利用\ref{root}式计算$x_1,x_2$};
		\node[draw, io , below=28pt of choice2](step 4){输出$x_1,x_2$的值};
		\node[draw, io , below=20pt of choice1](step 3){输出:该方程\\不存在实数根};
		\node[draw, rounded corners, below=of step 3](end1){结束};
		\node[draw, rounded corners, below=25pt of step 4](end2){结束};
				
		\graph{
			(start) -> (step 1) -> (step 2) -> (choice1) ->["否"left] (step 3) -> (end1);
			(choice1) ->["是"] (choice2) -> (step 4) -> (end2);
		};
\end{tikzpicture}
\caption{一元二次方程求解算法流程图}
\end{figure}
\section{用Gnuplot绘制方程解情况的示意图}
\begin{figure}[H]
	\includegraphics[width=.9\textwidth]{./qeou.png}
	\caption{一元二次方程三种接的情况示意图}
\end{figure}
	
\bibliographystyle{plain}
\bibliography{qeou.bib}	
	
\end{document}
