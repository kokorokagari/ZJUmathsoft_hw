%!Tex Program = xelatex
%\documentclass[a4paper]{article}
\documentclass[a4paper]{ctexart}

\usepackage{xltxtra}
\usepackage[utf8]{inputenc}%我也不知道有啥用
\usepackage{amsmath}
\usepackage{graphicx}
\usepackage{ctex}
\usepackage[T1]{fontenc}
\usepackage{textcomp}
\usepackage{url}
\usepackage{hyperref}

\usepackage{listings}
\usepackage{color}
\definecolor{dkgreen}{rgb}{0,0.6,0}
\definecolor{gray}{rgb}{0.5,0.5,0.5}
\definecolor{mauve}{rgb}{0.58,0,0.82}

\lstset{frame=tb,
	language=Python,
	aboveskip=3mm,
	belowskip=3mm,
	showstringspaces=false,
	columns=flexible,
	basicstyle={\small\ttfamily},
	numbers=left,%设置行号位置none不显示行号
	%numberstyle=\tiny\courier, %设置行号大小
	numberstyle=\tiny\color{gray},
	keywordstyle=\color{blue},
	commentstyle=\color{dkgreen},
	stringstyle=\color{mauve},
	breaklines=true,
	breakatwhitespace=true,
	escapeinside=``,%逃逸字符(1左面的键),用于显示中文例如在代码中`中文...`
	tabsize=4,
	extendedchars=false %解决代码跨页时,章节标题,页眉等汉字不显示的问题
}

\hypersetup{hypertex=true,
	colorlinks=true,
	linkcolor=blue,
	anchorcolor=blue,
	citecolor=blue}
%%%%%%%%%%%%%%%%%%%%%%%%%%%%%%%%%%%%%%%%%%%%%%%%%%%%%%%%%%%%%%%%%%%%%
%% Place any additional macros here.  Please use \newcommand* where
%% possible, and avoid layout-changing macros (which are not used
%% when typesetting).
%%%%%%%%%%%%%%%%%%%%%%%%%%%%%%%%%%%%%%%%%%%%%%%%%%%%%%%%%%%%%%%%%%%%%
%\newcommand*\mycommand[1]{\texttt{\emph{#1}}}

%\setmainfont[Mapping=tex-text]{AR PL UMing CN:style=Light}
%\setmainfont[Mapping=tex-text]{AR PL UKai CN:style=Book}
%\setmainfont[Mapping=tex-text]{WenQuanYi Zen Hei:style=Regular}
%\setmainfont[Mapping=tex-text]{WenQuanYi Zen Hei Sharp:style=Regular}
%\setmainfont[Mapping=tex-text]{AR PL KaitiM GB:style=Regular} 
%\setmainfont[Mapping=tex-text]{AR PL SungtiL GB:style=Regular} 
%\setmainfont[Mapping=tex-text]{WenQuanYi Zen Hei Mono:style=Regula}

\author{吴泓鹰\\数学与应用数学(强基计划)\quad3210101890}
\title{Mandelbrot Set的生成和探索}
%\date{}

\begin{document}
\maketitle	

\begin{abstract}
	%\centering%使得关键字居中
	\textbf{关键词:}Mandelbrot Set,Python
\end{abstract}	

Mandelbrot Set是什么东西呢?Mandelbrot Set相信大家都很(并不)熟悉,但是Mandelbrot Set到底是什么东西呢,下面就让小编带大家一起了解吧。

\section{问题背景}
Mandelbrot Set是一个几何图形,曾被称为“上帝的指纹”。 这个点集均出自公式$z_{n+1}={z_n}^2+c$,对于非线性迭代公式$z_{n+1}={z_n}^2+c$,所有使得无限迭代后的结果能保持有限数值的复数$z$的集合(也称该迭代函数的Julia集)连通的$c$,构成Mandelbrot Set。它是Mandelbrot教授在二十世纪七十年代发现的。\cite{mandelbrotsetbaidu}

迭代函数理论源于现实问题。种群增长的建模就是个例子。种群的当前规模决定了一个繁殖周期之后的规模。因此,种群增长的数学模型可以用一个包含自变量$x$的函数进行描述。$x$代表当前种群规模,$f(x)$代表一个繁殖周期后的期望种群规模。要想算出多个繁殖周期后的种群规模,就需要迭代该函数。由此,生活实践引发了人们对迭代理论的研究。

这引出了迭代理论涉及的重要问题之一:典型迭代轨迹的运动趋势如何?收敛还是发散?周期循环还是毫无规律可言?Mandelbrot Set对此问题做了图形式的解答。\cite{mandelbrotsetzhihu}

\section{生成与探索}
在本节我们将介绍Mandelbrot Set生成的数学力量,并且使用迭代算法对Mandelbrot Set进行绘制,得到其在不同条件下的.png图片。具体而言,我将使用Python进行编程以实现以上要求(源代码见\href{run:./code/py/mandelbrot.txt}{mandelbrot.txt})\cite{mandelbrotsetpython}。
\subsection{数学理论}

\subsection{基本算法}

\subsection{数值算例}

\section{结论与思考}
	
	
	
\bibliographystyle{plain}
\bibliography{mandelbrot.bib}
	
\end{document}