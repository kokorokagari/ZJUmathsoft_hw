%!Tex Program = xelatex
%\documentclass[a4paper]{article}
\documentclass[a4paper]{ctexart}
\usepackage{xltxtra}
%\setmainfont[Mapping=tex-text]{AR PL UMing CN:style=Light}
%\setmainfont[Mapping=tex-text]{AR PL UKai CN:style=Book}
%\setmainfont[Mapping=tex-text]{WenQuanYi Zen Hei:style=Regular}
%\setmainfont[Mapping=tex-text]{WenQuanYi Zen Hei Sharp:style=Regular}
%\setmainfont[Mapping=tex-text]{AR PL KaitiM GB:style=Regular} 
%\setmainfont[Mapping=tex-text]{AR PL SungtiL GB:style=Regular} 
%\setmainfont[Mapping=tex-text]{WenQuanYi Zen Hei Mono:style=Regula}

\title{作业二:Shell脚本的测试结果}
\author{吴泓鹰\\数学与应用数学(强基计划)\qquad3210101890}
\date{2022年6月28日}

\begin{document}
\maketitle

\section{Shell脚本的源码与测试}
在源码的编写上,我以教材\footnote{N. Matthew and R. Stones, Beginning Linux Programming,
	4th Edition, Wiley Publishing, Inc.,
	Indianapolis, 2008.
}上第40页的
\begin{verbatim}
	Try It Out Case IIl: Executing Multiple Statements
\end{verbatim}
为模板,进行了简单的本地化改动得到了本报告所展示的Shell脚本。
\subsection{源码展示}
\begin{verbatim}
	#!/bin/sh
	yes_or_no() {
		echo "你的名字是 $* 吗 ?"
		while true #这里的"true"可以用":"代替
		do
		echo -n "请输入 yes 或者 no: "
		read x
		case "$x" in
		y | yes | Y | Yes) return 0;;
		[Nn]* ) return 1;;
		* ) echo "请回答 yes 或者 no!"
		esac
		done
	}
	echo "这个程序的用户有 $*"
	if yes_or_no "$1"
	then
	echo "你好呀 $1, 真是个好名字~"
	else
	echo "打扰了~"
	fi
	exit 0
\end{verbatim}
\subsection{测试结果}
(a)输入:
\begin{verbatim}
	./shell_hw why yhw
	Y
\end{verbatim}
输出:
\begin{verbatim}
	why@why-virtual-machine:~/mathsoft/homework2$ ./shell_hw why yhw
	这个程序的用户有 why yhw
	你的名字是 why 吗 ?
	请输入 yes 或者 no: Y
	你好呀 why, 真是个好名字~
\end{verbatim}

(b)输入:
\begin{verbatim}
	./shell_hw why
	Noooo!
\end{verbatim}
输出:
\begin{verbatim}
	why@why-virtual-machine:~/mathsoft/homework2$ ./shell_hw why
	这个程序的用户有 why
	你的名字是 why 吗 ?
	请输入 yes 或者 no: Noooo! 
	打扰了~
\end{verbatim}

(c)输入:
\begin{verbatim}
	/shell_hw why yhw
	?
	...
	爬
	nanodesu
\end{verbatim}
输出:
\begin{verbatim}
	why@why-virtual-machine:~/mathsoft/homework2$ ./shell_hw why yhw
	这个程序的用户有 why yhw
	你的名字是 why 吗 ?
	请输入 yes 或者 no: ?
	请回答 yes 或者 no!
	请输入 yes 或者 no: ...
	请回答 yes 或者 no!
	请输入 yes 或者 no: 爬
	请回答 yes 或者 no!
	请输入 yes 或者 no: nanodesu
	打扰了~
\end{verbatim}
\section{Shell脚本的解析与学习}
在这一部分,我将对上述脚本的源码与输出结果进行解析,从而展示我在今天的课堂与课外中所学习到的有关Shell脚本的内容。
\subsection{源码与输出结果的解析}
{\bf 我们从将对源码上往下进行解析:}
\begin{itemize}
\item 在源码中的第一行与第四行都使用到了\verb|#|,但实际上他们的功能并不相同;第一行中\verb|#!/bin/sh|是一种特殊的注释,\verb|#!|告诉了系统应该执行\verb|/bin/sh|这个Shell程序;而第四行中的\verb|#|仅仅是告诉系统其后的内容属于注释,不需要执行。
\item 第二行中的\verb|yes_or_no(){...}|是一个函数,它可以被后面的程序所调用执行\verb|{...}|中的内容。
\item 源码中的\verb|echo|可以用于输出其后的变量内容(需要在变量名前加上\verb|$|)或者字符串,在\verb|echo|后加上\verb|-n|可以使输出后不自动换行。
\item while循环语句
\begin{verbatim}
	while 循环条件 do
	循环内容
	done
\end{verbatim}
正如上述代码展示的一样,在满足循环条件的情况下重复的执行循环内容;在源码中,条件\verb|true|即验证返回值是否为真,其中的\verb|return 0|表明返回值为假,\verb|return 1|表明返回值为真。
\item 源码中的\verb|read|可以用于读取用户所输入的一个字符串。
\item case条件语句
\begin{verbatim}
	case 变量 in
	条件 ) 执行内容;;
	条件 ) 执行内容;;
	...
	esac
\end{verbatim}
正如上述代码展示的一样,从上到下依次对变量验证条件,在变量满足条件的时候执行其后的内容,注意:这里的两个双引号\verb|;;|是必要的。
\item if条件语句
\begin{verbatim}
	if 条件
	then
		内容1
	else
		内容2
	fi
\end{verbatim}
正如上述代码展示的一样,验证条件是否成立,是则输出内容1,否则输出内容2。
\item 源码中的\verb|exit|将确保脚本返回一个有意义的退出码,一般而言退出码0表示程序成功运行,其他退出码则会有不同的含义。
\end{itemize}

{\bf 对测试结果的分析:}
\begin{itemize}
\item 注意到我们在执行文件\verb|./shell_hw|后加入了字符\verb|why|与\verb|yhw|,这将在脚本中作为参数变量出现,即源码中出现的\verb|$*,$1|,它们分别表示脚本程序的所有参数与第一个参数,因此在测试结果(a)中分别出现了\verb|why yhw|于\verb|why|。
\item 在测试结果(b)与(c)中我们发现输入\verb|Noooo!,nanodesu|也能输出,这是因为\verb|[]*|可以将只要包含方括号中内容的输入识别为条件成立。
\end{itemize}

\end{document}

