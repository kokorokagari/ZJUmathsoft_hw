\documentclass{ctexart}

\usepackage{graphicx}
\usepackage{amsmath}
\usepackage{amsthm}

\title{作业一: L'Hopital法则的叙述与证明}

\author{吴泓鹰 \\ 数学与应用数学(强基计划) 3210101890}

\begin{document}

\maketitle


这是一个来自分析学领域的问题,众所周知,两个无穷小之比或两个无穷大之比的极限可能存在,也可能不存在。因此,求这类极限时往往需要适当的变形,转化成可利用极限运算法则或重要极限的形式进行计算。洛必达法则便是应用于这类极限计算的通用方法。
\section{问题描述}
问题叙述如下: 函数比值的极限等于其导数比值的极限,只要导数的比值的极限存在。

用数学语言则表示为:

(a)如果函数$f(x)$与函数$g(x)$在去心邻域$\tilde{B}_{\delta}(x)$内可导且$g(x)\ne0$恒成立,满足:
\begin{equation}
	\lim_{x \rightarrow a}f(x)=\lim_{x \rightarrow a}g(x)=0;\qquad\lim_{x \rightarrow a}\frac{f'(x)}{g'(x)}=l;\label{1}
\end{equation}
则必有:
\begin{equation}
	\lim_{x \rightarrow a}\frac{f(x)}{g(x)}=l.\label{2}
\end{equation}

(b)如果函数$f(x)$与函数$g(x)$在去心邻域$\tilde{B}_{\delta}(x)$内可导且$g(x)\ne0$恒成立,满足:
\begin{equation}
	\lim_{x \rightarrow a}g(x)=\infty;\qquad\lim_{x \rightarrow a}\frac{f'(x)}{g'(x)}=l;\label{3}
\end{equation}
则必有:
\begin{equation}
	\lim_{x \rightarrow a}\frac{f(x)}{g(x)}=l.\label{4}
\end{equation}

注:这里的$a,l$可为有限的或者$\infty$。
\section{证明}
\begin{proof}[证明]
对于完整的证明而言,我们需要考虑4种极限过程:分别是$x \rightarrow a^-,x \rightarrow a^+,x \rightarrow -\infty,x \rightarrow +\infty$,这看上去十分复杂,但实际上,对于这4种极限过程,定理的证明是完全类似的,我们只需要对其中$x \rightarrow a^+$的情形详细地写出证明,并简要的叙述出其他3种情形下证明应作出的改动。

我们将只对$\displaystyle \lim_{x \rightarrow a}\frac{f'(x)}{g'(x)}=l$为有限的情形进行证明,因为当$l=+\infty$或者$l=-\infty$时,证明同样是完全类似的。

首先我们对(a)进行证明,由式\ref{1}与极限定义可知:对于$\forall\epsilon>0,\exists\delta>0$,当$x\in(a,a+\delta)$时,有:
$$
	l-\epsilon<\frac{f'(x)}{g'(x)}<l+\epsilon;
$$
因此对于任意子区间$(x,x_0)\subset(a,a+\delta)$,由Cauchy中值定理,必然$\exists\xi\in(x,x_0)$,使得:
\begin{equation}
	l-\epsilon<\frac{f(x)-f(x_0)}{g(x)-g(x_0)}=\frac{f'(\xi)}{g'(\xi)}<l+\epsilon;\label{5}
\end{equation}
通过恒等变形可以得到:
$$
\frac{f(x)-f(x_0)}{g(x)-g(x_0)}=\frac{\frac{f(x)}{g(x)}-\frac{f(x_0)}{g(x)}}{1-\frac{g(x_0)}{g(x)}};
$$
在上式中令$x_0\rightarrow a$,有$1-\displaystyle\frac{g(x_0)}{g(x)}\rightarrow 1$,从而我们得到:$\displaystyle\varlimsup_{x \rightarrow a^+}\frac{f(x)}{g(x)}\le l+\epsilon$.
由$\epsilon$的任意性,我们得到$\displaystyle\varlimsup_{x \rightarrow a^+}\frac{f(x)}{g(x)}\le l$,同理有$\displaystyle\varliminf_{x \rightarrow a^+}\frac{f(x)}{g(x)}\ge l$.
故由上述两式有式\ref{2}成立。

接下来我们证明情形(b),由恒等变形我们有:(这里$x$与$x_0$的定义与上述证明中的一致)
\begin{equation}
\begin{aligned}
&\frac{f(x)}{g(x)}=\frac{f(x)-f(x_0)}{g(x)}+\frac{f(x_0)}{g(x)}\\
&=\frac{g(x)-g(x_0)}{g(x)}\cdot\frac{f(x)-f(x_0)}{g(x)-g(x_0)}+\frac{f(x_0)}{g(x)}\\
&=\left(1-\frac{g(x_0)}{g(x)}\right)\frac{f(x)-f(x_0)}{g(x)-g(x_0)}+\frac{f(x_0)}{g(x)}\nonumber
\end{aligned}
\end{equation}
于是,我们有:
\begin{equation}
	\begin{aligned}
		&\left|\frac{f(x)}{g(x)}-l\right|=\left|\left(1-\frac{g(x_0)}{g(x)}\right)\frac{f(x)-f(x_0)}{g(x)-g(x_0)}+\frac{f(x_0)}{g(x)}-l\right|\\
		&\le\left|1-\frac{g(x_0)}{g(x)}\right|\cdot\left|\frac{f(x)-f(x_0)}{g(x)-g(x_0)}-l\right|+\left|\frac{f(x_0)-lg(x_0)}{g(x)}\right|\label{6}
	\end{aligned}
\end{equation}
其中由Cauchy中值定理可得:(这里$\xi$的定义与上述证明中的一致)
$$
\left|\frac{f(x)-f(x_0)}{g(x)-g(x_0)}-l\right|=\left|\frac{f'(\xi)}{g'(\xi)}-l\right|<\epsilon
$$
结合式\ref{3}可知当$x$趋于$a$时,有$\displaystyle\left|1-\frac{g(x_0)}{g(x)}\right|$小于常数,且$\displaystyle\left|\frac{f(x)-f(x_0)}{g(x)-g(x_0)}-l\right|$趋于$0$,又$\displaystyle\left|\frac{f(x_0)-lg(x_0)}{g(x)}\right|$的分母趋于$\infty$,分子是固定的,所以也趋于$0$。
综上所述有:
\begin{equation}
	\lim_{x \rightarrow a}\frac{f(x)}{g(x)}=l=\lim_{x \rightarrow a^+}\frac{f'(x)}{g'(x)}\nonumber
\end{equation}
即式\ref{4}成立。

注:在其余的3种情形中,只需要更改$x,x_0$的取值范围即可完成类似的完成证明。
\end{proof}
\end{document}
