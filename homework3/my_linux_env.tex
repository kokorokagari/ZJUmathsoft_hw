%!Tex Program = xelatex
%\documentclass[a4paper]{article}
\documentclass[a4paper]{ctexart}
\usepackage{xltxtra}
\usepackage{url}
\usepackage{hyperref}
%\setmainfont[Mapping=tex-text]{AR PL UMing CN:style=Light}
%\setmainfont[Mapping=tex-text]{AR PL UKai CN:style=Book}
%\setmainfont[Mapping=tex-text]{WenQuanYi Zen Hei:style=Regular}
%\setmainfont[Mapping=tex-text]{WenQuanYi Zen Hei Sharp:style=Regular}
%\setmainfont[Mapping=tex-text]{AR PL KaitiM GB:style=Regular}
%\setmainfont[Mapping=tex-text]{AR PL SungtiL GB:style=Regular}
%\setmainfont[Mapping=tex-text]{WenQuanYi Zen Hei Mono:style=Regular}
\hypersetup{colorlinks=true,linkcolor=black}

\title{作业三:我的linux工作环境}
\author{吴泓鹰\\数学与应用数学(强基计划)\qquad 3210101890}

\begin{document}
\maketitle
Linux系统作为一个切实可行的操作系统,其作为一个类UNIX的内核实现,具有{\bf 简单性、集中性、可重用组件、过滤器、开放的文件格式、灵活性}等特点\cite{linux_book},因此到了今天也依旧是我们学习与工作的良好选择。因此,我通过学习书籍与查找网页资料\cite{linux_book}\cite{linux_cainiao},搭建了一个属于自己的Linux工作环境,并在此进行简单介绍。
\section{Linux系统的安装}
作为一名大一新生,在学习该门短学期课程之前一直使用的是Windows操作系统,因此我需要在原有的Windows系统的基础上安装Linux系统,通过查询资料\cite{linux_anz}了解到可以通过使用双系统(WSL\footnote{适用于 Linux 的 Windows 子系统可让开发人员按原样运行 GNU/Linux 环境 - 包括大多数命令行工具、实用工具和应用程序 - 且不会产生传统虚拟机或双启动设置开销。})或者虚拟机进行Linux系统的安装,处于本人的水平考虑,我选择了使用虚拟机安装Linux系统。
\subsection{使用虚拟机安装Linux系统}\label{1.1}
通过查询资料\cite{linux_ubuntu},虚拟机方面,我选择了使用VMware Workstation 16 Pro进行Linux系统的安装,并且使用了Ubuntu 22.04作为桌面系统,对其进行了如下的具体设置:

\begin{tabular}{l|l}
	\hline
	设备 & 摘要\\
	\hline
	内存 & 4 GB \\
	处理器 & 2 \\
	硬盘(SCSI) & 40 GB,拆分 \\
	网络适配器 & NAT \\
	其他设备 & CD/DVD,声卡 \\
	\hline
\end{tabular}
\subsection{Linux发行版名称与版本号}
在键盘上同时按下\verb|CTRL+ALT+T|打开终端页面,在其中输入\verb|lsb_release -a|命令即可得到Linux发行版名称与版本号,输出如下:
\begin{verbatim}
	No LSB modules are available.
	Distributor ID:	Ubuntu
	Description:	Ubuntu 22.04 LTS
	Release:	22.04
	Codename:	jammy
\end{verbatim}
\section{系统调整、软件安装与配置工作}
为了更好地使用Linux系统,需要对它的工作环境进行一定的配置与调整,通过查找资料\cite{linux_ubuntu}\cite{linux_set},我对其进行了以下配置与调整。
\subsection{系统调整}
正如\ref{1.1}中所展示的那样,我只对系统硬件进行了一些调整(如删除打印机、USB接口,修改内存大小等),并未进行更为深入的系统调整。
\subsection{软件安装}
为了更为方便以及个人需要方面的考虑,我安装了以下软件进行使用:
\begin{itemize}
	\item 输入法:Fcitx 4键盘输入法系统,Google输入法
	\item 编译器:gcc,g++
	\item 编译(安装)工具:make,cmake,automake
	\item 编辑器 emacs,vim,gedit,TeXstudio
	\item 文档工具:texlive,doxygen
	\item 数字计算处理包:libboost,trilinos
	\item 后处理:dx
	\item 源代码管理:git
	\item 远程管理:ssh,vnc-server,x11vnc
	\item 软件管理:Synaptic
	\item 浏览器:Chrome
	\item 桌面管理:GNOME-Tweaks
	\item pdf阅读器:Okular
\end{itemize}
\subsection{配置工作}
整体而言,我主要对{\bf git,vim以及软件源}的配置进行了如下的修改:
\begin{enumerate}
	\item 配置了git的全局配置,包括用户\verb|[user]|与代理\verb|[http][https]|,并且使用了ssh密钥来方便源代码的\verb|push|与\verb|clone|。
	\item 在用户目录\verb|~\|下添加了配置文件\verb|.vimrc|以增加自己的学习与工作效率(具体配置见参考文献\cite{linux_set})
	\item 在应用“软件与更新”中修改了软件源为\url{http://mirrors.huaweicloud.com/repository/ubuntu}
\end{enumerate}
\section{对下一步工作的规划}

\subsection{未来半年内可能的Linux使用场合}
数学专业课程{\bf 数据结构与算法},使用{\bf \LaTeX}编写数学相关文档,使用{\bf 终端}更为方便地进行文件处理相关的事项。
\subsection{对目前工作环境的分析与改动计划}
就目前而言,我仍对自己的工作环境的整体情况不够了解,仍需要更多的学习与实践,无法给出有效的工作环境分析与改动计划。
\section{工作结果稳定与安全的保障}
众所周知,文件、文档、源代码等都是我们重要的工作结果,一旦缺失或者丢失将会造成难以预计的后果,以至于对我们的工作与生活造成巨大影响。因此,保障我们工作成功的稳定与安全就十分必要了。

一般而言,我会使用git进行源代码管理并将代码上传至{\bf github},从而在云端保存好自己的源代码;而对于相对体积较大的文档而已,我会选择使用{\bf 坚果云}(\url{https://www.jianguoyun.com/})来保存自己的重要文档,以免丢失。

\bibliographystyle{plain}
\bibliography{my_linux_env.bib}

\end{document}
