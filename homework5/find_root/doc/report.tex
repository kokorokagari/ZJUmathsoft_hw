%!Tex Program = xelatex
%\documentclass[a4paper]{article}
\documentclass[a4paper]{ctexart}

\usepackage[utf8]{inputenc}%告诉 LaTeX TeX文档的输入(文件)编码(为utf-8)
\usepackage{amsmath}%数学公式相关
\usepackage{amssymb}%更多的数学符号
\usepackage{caption}%控制题目/说明文字的样式
\usepackage{graphicx, subfig}%插入图片
\usepackage{ctex}%中文宏包
\usepackage{float}%强行保留图片位置用
%\usepackage[T1]{fontenc}%可以调用各种编码字体,并可有多重选项,其中最后的选项将成为系统默认编码
%\usepackage{textcomp}%提供了许多 TS1 编码的可直接在文本中使用的常见数理单位和货币符号
\usepackage{url}%插入网址
\usepackage{hyperref}%超链接

\usepackage{listings}%更好看的代码
\usepackage{xltxtra}%listings中文支持
\usepackage{color}%listings字体颜色相关
\definecolor{dkgreen}{rgb}{0,0.6,0}
\definecolor{gray}{rgb}{0.5,0.5,0.5}
\definecolor{mauve}{rgb}{0.58,0,0.82}

\lstset{frame=tb,%listings代码区各种设置
	language=Python,
	aboveskip=3mm,
	belowskip=3mm,
	showstringspaces=false,
	columns=flexible,
	basicstyle={\small\ttfamily},
	numbers=left,%设置行号位置none不显示行号
	%numberstyle=\tiny\courier, %设置行号大小
	numberstyle=\tiny\color{gray},
	keywordstyle=\color{blue},
	commentstyle=\color{dkgreen},
	stringstyle=\color{mauve},
	breaklines=true,
	breakatwhitespace=true,
	escapeinside=``,%逃逸字符(1左面的键),用于显示中文例如在代码中`中文...`
	tabsize=4,
	extendedchars=false %解决代码跨页时,章节标题,页眉等汉字不显示的问题
}

\hypersetup{hypertex=true,%超链接设置
	colorlinks=true,
	linkcolor=blue,
	anchorcolor=blue,
	citecolor=blue}
%%%%%%%%%%%%%%%%%%%%%%%%%%%%%%%%%%%%%%%%%%%%%%%%%%%%%%%%%%%%%%%%%%%%%
%% Place any additional macros here.  Please use \newcommand* where
%% possible, and avoid layout-changing macros (which are not used
%% when typesetting).
%%%%%%%%%%%%%%%%%%%%%%%%%%%%%%%%%%%%%%%%%%%%%%%%%%%%%%%%%%%%%%%%%%%%%
%\newcommand*\mycommand[1]{\texttt{\emph{#1}}}

%\setmainfont[Mapping=tex-text]{AR PL UMing CN:style=Light}
%\setmainfont[Mapping=tex-text]{AR PL UKai CN:style=Book}
%\setmainfont[Mapping=tex-text]{WenQuanYi Zen Hei:style=Regular}
%\setmainfont[Mapping=tex-text]{WenQuanYi Zen Hei Sharp:style=Regular}
%\setmainfont[Mapping=tex-text]{AR PL KaitiM GB:style=Regular} 
%\setmainfont[Mapping=tex-text]{AR PL SungtiL GB:style=Regular} 
%\setmainfont[Mapping=tex-text]{WenQuanYi Zen Hei Mono:style=Regula}

\author{吴泓鹰\\数学与应用数学(强基计划)\quad3210101890}
\title{\textbf{gsl Example:}\\ \texttt{roots.c}的功能分析}
%\date{}

\begin{document}
\maketitle

\section{功能分析}
\textbf{简单版}:该代码使用了布伦特方法(Brent's method)\footnote{布伦特方法是在二分法或试位法的基础上,借助二次插值方法进行加速,有利用反插值方法来简化计算而形成的一种方法,详见\href{https://en.wikipedia.org/wiki/Brent's_method}{wikipedia}\label{web}}来求解函数$f(x)=x^2-5$的根(roots),输出结果为$x=\sqrt5$的估计之。

\textbf{复杂版}:该代码设定初始化区间$[a_0,b_0]$使得$f(a_0)\cdot f(b_0)<0$,通过$k$次迭代得到区间$[a_k,b_k]$使得$f(a_k)\cdot f(b_k)<0$,其中$a_k,b_k$分别是上一次迭代中根的估计值与$a_{k-1},b_{k-1}$中使得满足条件$f(a_k)\cdot f(b_k)<0$的一项(可能顺序反过来),当判断得到的估计值满足精度要求(这里是$|f|<0.001$)时停止迭代,输出结果。迭代法求根的估计值一般使用二分法、割线法(线性插值)和逆二次插值法,通过某种方法\textsuperscript{\ref {web}}判断出最优选择使用其中的一个完成根的估计值的求取。

%\section{源码与分析}

\section{输出结果}
在终端输入\texttt{make}后在\texttt{.$\backslash$bin}目录下生成可执行文件\texttt{roots},在终端输入\texttt{.$\backslash$bin$\backslash$roots}即可得到如下所示的输出结果:
\begin{verbatim}
using brent method
iter [    lower,     upper]      root        err  err(est)
1 [1.0000000, 5.0000000] 1.0000000 -1.2360680 4.0000000
2 [1.0000000, 3.0000000] 3.0000000 +0.7639320 2.0000000
3 [2.0000000, 3.0000000] 2.0000000 -0.2360680 1.0000000
4 [2.2000000, 3.0000000] 2.2000000 -0.0360680 0.8000000
5 [2.2000000, 2.2366300] 2.2366300 +0.0005621 0.0366300
Converged:
6 [2.2360634, 2.2366300] 2.2360634 -0.0000046 0.0005666
\end{verbatim}

第一列表示迭代次数,第二列第三列分别表示$a_k,b_k$,第四列表示上一次迭代求出来的根的估计值,第五列第六列分别表示与$\sqrt5$的误差和上一次估计值的误差,\verb|Converged:|后表示最终结果。
\end{document}	
